%!TEX program = xelatex
\documentclass[UTF8,11pt]{ctexart}
\usepackage{amsthm, amsfonts, amsmath, geometry}
\usepackage{graphicx}
\geometry{scale=0.8}
\title{数学分析与线性代数例题}
\author{佚名}
\date{2019年12月6日}
\newtheorem{theorem}{定理}
\newtheorem{exmp}{例}
\begin{document}

\maketitle{}
\tableofcontents


\section{微分中值定理及其应用}
\begin{theorem}[极限的第二充分条件]设 $f(x)$ 在 $(x_0-\delta,x_0+\delta)$ 可导且 $f'(x_0)=0$,又 $f''(x)$ 存在.

\vspace{5mm}
1)若 $f''(x_0)<0$ ,则 $f(x_0)$ 是严格极大值;

\vspace{5mm}
2)若 $f''(x_0)>0$ ,则 $f(x_0)$ 是严格极小值.

\end{theorem}
\begin{exmp}
\kaishu 求 $y=\dfrac{1}{3}x\sqrt[3]{(x-5)^2}$ 的极值点与极值\footnote{原题摘自《数学分析简明教程》(上册)P142.}.
\end{exmp}
\leftline {\heiti 解. \kaishu 函数在 $(-\infty,+\infty)$ 上连续,当 $x \not = 5$ 时有}
\begin{equation}
\label{eq:prob}
y'=\dfrac{1}{3}\Big({(x-5)}^{\frac{2}{3}}+\dfrac{2x}{3}(x-5)^{-\frac{1}{3}}\Big)=\dfrac{5(x-3)}{9(x-5)^{1/3}}
\end{equation}
\leftline {\kaishu 令 $y'=0$ 得稳定点 $x=3$ ,现列表如下:}\\
\vspace{-6mm}
\begin{center}
\begin{tabular}{|c|c|c|c|c|c|}\hline
$x$&$(-,3)$&$3$&$(3,5)$&$5$&$(5,+)$\\\hline
$y'$&$+$&$0$&$-$&不存在&$+$\\\hline
$y$&$\nearrow$&$\sqrt[3]{4}$&$\searrow$&$0$&$\nearrow$\\\hline
\end{tabular}\\
\end{center}

\kaishu 从表中可见 $x = 3$ 是极大值点,极大值为 $f(3) = \sqrt[3]{4};x = 5$ 为极小值点,极小值为 $f(5) = 0$. 我们可以大致地画出函数的图形,如图\ref{fig:function}所示.
\newpage
\begin{figure}[ht]
\begin{center}
\includegraphics[width=9cm]{function.pdf}
\caption{\small\it  $y=\dfrac{1}{3}\sqrt[3]{(x−5)^2}$  的函数图像}
\label{fig:function}
\end{center}
\end{figure}

\section{行列式}
\begin{exmp}
\kaishu 若 $a,b\in \mathbb{R}^2$, 求由方程$\dfrac{x^2_1}{a^2}+\dfrac{x_2^2}{b^2}=1$ 的椭圆为边界的区域$E$的面积\footnote{原题摘自《线性代数及其应用》(第三版)P183.}\\
\heiti 解. \kaishu 断言 $D$ 是单位圆盘 $D$ 在线性变换 $T$ 下的像. 这里 $T$ 由矩阵 $A = \begin{bmatrix}
a & 0 \\
0 & b
\end{bmatrix}
$ 确定,这是因为若$\mathbf{u} = \begin{bmatrix} u_1\\u_2 \end{bmatrix}$, $\mathbf{x} = \begin{bmatrix} x_1\\x_2 \end{bmatrix}$, 且$\mathbf{x} = A\mathbf{u}$,则\\
\begin{center}
$u_1=\dfrac{x_1}{a},u_2=\dfrac{x_2}{b}$\\
\end{center}
从而得 $\mathbf{u}$ 在此单位圆内,即满足 $u^2_1 + u^2_2 \leq 1$,当且仅当 $\mathbf{x}$ 在 $E$ 内,即满足 $(x_1/a)^2 + (x_2/b)^2 \leq 1$. 进而\\
\begin{align}
\notag
\{\text{椭圆的面积}\} &= \{T(D)\text{的面积}\}\\\notag
&= \left|detA\right|·\{D\text{的面积}\}\\\notag
&= a·b·\pi·(1)^2\\\notag
&= \pi ab
\end{align}
\end{exmp}
\end{document}